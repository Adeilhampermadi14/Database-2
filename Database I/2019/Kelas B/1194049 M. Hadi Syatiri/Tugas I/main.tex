\documentclass{article}

\title{Rangkuman Basis Data}
\author{M.Hadi Syatiri}
\date{25 February 2020}

\usepackage[left=2.00 cm, bottom=4.00 cm, right=2.00 cm, top=3.00 cm]{geometry}

\begin{document}

\maketitle

\section{Basis Data}
\maketitle
Basis Data terdiri dari dua kata yaitu basis dan data.
\begin{itemize}
    \item Basis : Kumpulan, Bidang, Gabungan, Tempat.
    \item Data : Informasi, Angka, Gambar, Simbol, Nilai.
\end{itemize}
Jadi BASIS DATA adalah kumpulan data-data yang disimpan di media teknologi sehingga bisa dicari dan diakses. Basis data juga bisa kita temukan di kehidupan kita sehari-hari seperti lemari,kulkas,KTP,tas,dll. Kita ambil contoh lemari, di dalem lemari terdapat baju,celana,dan lain-lain. Basis Data harus melewati tahapan normalisasi dan redudansi agar tidak terjadi pengulangan, contoh nama mahasiswa (Syatiri, syatiri) agar tidak ada yang double harus melewati tahapan tersebut.

\section{Tujuan Basis Data}
\maketitle
Tujuan dari basis data ini adalah:
\begin{enumerate}
\item
Untuk memudahkan menyimpan data, melakukan perubahan terhadap data, dan menampilkan kembali data dengan lebih cepat dan mudah dibandingkan dengan cara manual.
\item
Untuk efisien ruang penyimpanan. Adanya basis data ini akan mengurangi kerangkapan data dan menghemat ruang untuk menyimpan data.
\item
Untuk menjaga keamanan data agar tidak semua orang bebas untuk mengakses data tersebut.
\end{enumerate}

\section{Database Management System}
\maketitle
Database Management System atau yang biasa disingkat DBMS merupakan sebuah software yang digunakan untuk mengorganisasikan suatu basis data. Ada beberapa jenis DBMS yang biasa digunakan, diantaranya:
\begin{itemize}
    \item MySQL
    \item Oracle
\end{itemize}

\end{document}
